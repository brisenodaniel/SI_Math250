\documentclass[12pt]{article}
\usepackage{mathtools}
\usepackage{amsfonts}
\usepackage{amssymb}
\usepackage{amsthm}
\usepackage{enumitem}
\usepackage[a4paper, total={6in, 10in}]{geometry}
\date{}
\author{}
\title{LCD, GCD, and Bezout's Theorem}
\begin{document}
	\maketitle
	\section{Euclid's Algorithm}
	\textbf{Euclid's Algorithm for finding GDC is as follows:}
	\begin{align*}
	gcd(0,n)&=n\\
	gcd(m,n)&=gcd(n\%m,m) &&\text{for any $m>0$}
	\end{align*}
	\textbf{Calculate the following GCD's using Euclid's Algorithm, show steps:}
	\begin{enumerate}
		\item $gcd(3,9)$
		\item $gcd(0,24)$
		\item $gcd(436,559)$
		\item $gcd(5,525)$
		\item $gcd(123456789,123456790)$
	\end{enumerate}
	\section{Proofs regarding gcd and lcm}
	\textbf{Recall the definition of $fib(n)$}
	\begin{align*}
		fib(0)&=0\\
		fib(1)&=1\\
		fib(k^{\curvearrowright \curvearrowright})&=fib(k^\curvearrowright)+fib(k)
	\end{align*}
	\textbf{Construct a table of the first 10 values for the following algorithm:}
	\begin{enumerate}[resume]
		\item$gcd(fib(n^\curvearrowright),fib(n))$ 
	\end{enumerate}
	\textbf{Prove the Following:}
	\begin{enumerate}[resume]
		\item $gcd(fib(n^\curvearrowright),fib(n))=1$ for any $n\in\mathbb{N}$
		\item $n \cdot m = gcd(n,m)\cdot lcm(m,n)$
	\end{enumerate}
\end{document}