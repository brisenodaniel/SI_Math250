\documentclass[12pt]{article}
\usepackage{amsmath}
\usepackage{amsfonts}
\usepackage{amssymb}
\usepackage[a4paper, total={6in, 10in}]{geometry}
\date{}
\author{}
\title{SI 1: Proof Writing and Induction}
\begin{document}
	\maketitle
	\section{Contrapositive}
	\paragraph{A Definition: } The contrapositive is a sort of ``negative" form of a statment ``if $A$ then $B$". That is, given a statment ``if $A$ then $B$", then that immediately implies the contrapositive, or ``if not $B$ then not $A$". This can be useful when presented with some statement which seems very difficult to prove directly, but whose contrapositive is easier to prove. Proving the contrapositive of a statment is equivalent to proving the original statment.
		\subsection{Find the contrapositive form for the following:}
		\begin{enumerate}
		\item If it is raining, my car is wet.
		\item If $n\neq m$, then $n+p\neq p+m$.
		\item If $n=m$, and $m=p$, then $n=p$.
		\item If a student is not sleep deprived during finals, they are not taking Math 250.
		\end{enumerate}
	\section{Definitions}
		\paragraph{Importance of Definitions: } As many of you may have already noticed, definitions are very important in proof writing. Even for something as simple and intuitive as $\leq$, we cannot reason about it unless we have made a precise definition for it.

		\subsection{Writing Definitions}
Write a definition for $\leq$ in $\mathbb{N}$. Hint: what exactly does $m\leq n$ mean $\forall n,m\in\mathbb{N}$?
		\subsection{Using Definitions}
		\paragraph{Prove: } $\forall m,n,p\in\mathbb{N}$, if $m\nleq n$, then either $m\nleq p$ or $p \nleq n$
		
		\newpage
	\section{Basic Proof Writing}
		\textbf{For the next two problems, you may use the following axioms:}
		\begin{enumerate}
			\item \textbf{Progression: } To take an $n^{th}$ step, we must begin at some step $s$ such that $s<n$ and $s\geq 0$, and end at the $n^{th}$ step
			\item \textbf{Counting: } $\forall m,n\in\mathbb{N}$ such that $m<p$, if we begin at counting at $m$, and count up to $p$, we have counted on the interval $(m,p)$
			\item\textbf{Skipping: } $\forall m,n,p\in\mathbb{N}$ such that $m<n<p$, to count from $m$ to $p$ without counting to n, means we have skipped at least one number $n$.
			\item\textbf{Linearity: } In a real-world progression of things, it is impossible to skip an item in the progression.
		\end{enumerate}
		\subsection{Lemma 1}
		\paragraph{Prove:} When walking from any point A to any point B, it is impossible to take a second step without first having taken a first step.
		\subsection{Proof of the Linearity of taking 3 steps}
		\paragraph{Prove by Induction:} $\forall n\in\mathbb{N}$, it is impossible to take an $n+1^{th}$ step without having taken an $n^{th}$ step.\newline
	\section{Some More Difficult Proofs}
		\subsection{Factorization of numbers greater than 1}
		\paragraph{Prove:} Every natural number $n>1$ is either prime or a product of primes
		\subsection{Towers of Hanoi}
		\paragraph{Prove:} For a tower of $n$ disks, it takes $2^n-1$ moves to solve the problem of the Towers of Hanoi.\newline
		\paragraph{The problem of the Towers of Hanoi:} There are $n$ disks stacked on one of 3 pegs, ordered from smallest disk to largest (smallest disk on the top). The task is to move all the disks to another peg following this set of rules:
		\begin{enumerate}
			\item Only one disk may be moved at a time
			\item No disk may be placed on a peg with a smaller disk underneath it
		\end{enumerate}
\end{document}