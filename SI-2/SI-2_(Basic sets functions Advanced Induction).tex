\documentclass[12pt]{article}
\usepackage{amsmath}
\usepackage{amsfonts}
\usepackage{amssymb}
\usepackage[a4paper, total={6in, 10.5in}]{geometry}
\date{}
\author{}
\title{SI Session 2: Basic Set Theory, Functions, and More Advanced Inductive Proof Writing}
\begin{document}
	\maketitle

\section{Set Theory}
	\subsection{Basic Grammar}
	\paragraph{Define: } $A =\{1,2,3,\{3,2\},2,4\};   B=\{1,2,3\}; C=\{A,\{B\},C\}$
	\begin{enumerate}
	\item True or false: $A \in B$
	\item True or false: $B \in A$
	\item True or false: $B \subseteq A$
	\item True or false: $\{A\}\in C$
	\item True or false: $\{B\} \in C$
	\item True or false: $\emptyset \in C$
	\item True or false: $\emptyset \subset C$
	\item Something here does not make sense, find it.
	\end{enumerate} 
	\subsection{Carteisan Product}
	\paragraph{Define: } $K=\{red, blue, green\}; L=\{1,2,3\}$
	\begin{enumerate}
	\item Calculate: $K \times L$
	\item Calculate: $L \times K$
	\item Calculate:$ |K|, |L|, |L\times K|$
	\end{enumerate}
\newpage
\section{Functions}
	\subsection {Domain, Co-Domain and Function Composition}
	\begin{enumerate}
	\item Define the behavior of a function that takes a $\mathbb{N}$ as input, and outputs into the set $A$, using the correct notation.
	\item Find the Domain and Co-domain: $f(x)=x^2$
	\item Define a function that has $\mathbb{I}$ as its domain, and $\mathbb{N}$ as its co-domain.
	\item Define a function that has $\mathbb{N}$ as its domain, and $\mathbb{R}$ as its co-domain.
	\item Define a function which can only have $\mathbb{N}$ as its domain.
	\item You are tasked with defining a function which encodes the location of a cell in a grid of pre-defined length. The length is given as an pair $(j,k)$, and you must encode it into a single natural number. Define the behavior of this function using function composition.
	\end{enumerate}
	\subsection{One-to-One and Onto Functions}
	\paragraph{For the next section:} Indicate whether the function is one to one, onto, both or neither
	\begin{enumerate}
	\item $f(x)=x^2$ for $f:\mathbb{R} \to \mathbb{R}$
	\item $f(x)=x^2$ for $f:\mathbb{C} \to \mathbb{I}$
	\item $f(x)=x$
	\item $f(x)=2^x$
	\item $f(n)= \sum_{k=1}^n k$
	\end{enumerate}
	\paragraph{Define the following: }
	\begin{enumerate}
	\item A function that is one to one but not onto
	\item A function that is onto but not one to one
	\item A function that is neither one to one or onto
	\item A function that is one to one and onto
	\end{enumerate}
\section{Induction}
	\textbf{Prove by induction: } \\ \small $a_1 b_1+...+a_n b_n=(a_1-a_2)b_1+(a_2-a_3)(b_1+b_2)+...+(a_{n-1} - a_n)(b_1+...+b_{n-1})+a_n(b_1+...+b_n)$
\end{document}