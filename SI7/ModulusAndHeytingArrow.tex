\documentclass[12pt]{article}
\usepackage{mathtools}
\usepackage{amsfonts}
\usepackage{amssymb}
\usepackage{amsthm}
\usepackage{enumitem}
\usepackage[a4paper, total={6in, 10in}]{geometry}
\date{}
\author{}
\title{Modular Arithmetic, Min, Max, and a Review of Strong Induction}
\begin{document}
	\maketitle
	\section{Modular Arithmetic}
	\textbf{For the next few problems you will need to recall a few facts of the Gregorian calendar}
	\\
	\\Leap years occur every 4th year, except for years ending in 00.\\
	The following months have 30 days: September, April, June, November\\
	February has 28 days on common years, 29 on leap years\\
	All other months have 31 days\\
	\\
	\textbf{Find the day of the week for the following dates}
	\begin{enumerate}
		\item 10/22/21
		\item 12/10/46
		\item 11/9/2118
	\end{enumerate}
	\section{Strong induction}
	\begin{enumerate}[resume]
		\item Show that for any natural numbers, $m$ and $m\cdot n +1$ are always relatively prime given $m\neq1$
	\end{enumerate}
	\section{Min and Max}
	\textbf{Consider the following definitions}
	\begin{align*}
	p\leq m \to n \:\:\:&\text{if and only if}\:\:\: min(p,m)\leq n\\
	m \:\theta\: n \leq p\:\:\: &\text{if and only if}\:\:\: m\leq max(n,p) 
	\end{align*}
	\begin{enumerate}[resume]
		\item Prove using the definitions of $\theta$ and $\to$: $max(m,min(n,p))=min(max(m,n),max(m,p))$
	\end{enumerate}
\end{document}