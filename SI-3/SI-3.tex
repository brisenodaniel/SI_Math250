\documentclass[12pt]{article}
\usepackage{mathtools}
\usepackage{amsfonts}
\usepackage{amssymb}
\usepackage{amsthm}
\usepackage[a4paper, total={6in, 10.5in}]{geometry}
\date{}
\author{}
\title{SI 3: Midterm 1 Review}
\begin{document}
\maketitle
\section{Sets}
\textbf{Consider the following sets:}
\begin{itemize}
\item $A: \{1,2,3,4,5\}$
\item $B: \{2,3,4\}$
\item $C: \{\{2\},\{3\},\{4\}\}$
\item $D: \{Red, Green, Blue\}$
\end{itemize}
\textbf{Define the Following:}
\begin{enumerate}
\item$B\times D$
\item$D\times B$
\item $ \mathcal{P}(B)$
\end{enumerate}
\textbf{Answer True or False for the following:}
\begin{enumerate}
\item$B\in A$
\item$B\subseteq A$
\item$C\subseteq A$
\item$A\subseteq A$
\item$A\in A$
\item$2\in C$
\item$A\in \mathbb{N}$
\item$A \subseteq \mathbb{N}$
\item$C \subseteq \mathbb{N}$
\item$\mathcal{P}(B)\in B$
\item$B \in \mathcal{P}(B)$
\item$B\subseteq\mathcal{P}(B)$
\item$\emptyset \in B$
\item$\emptyset \subseteq B$
\item$\emptyset \in \mathcal{P}(B)$
\item$\emptyset \subseteq \mathcal{P}(B)$
\end{enumerate}
\newpage
\section{Functions}
\subsection{Identifying Co-Domains, Domains, and basic \\function properties}
\textbf{Identify Domain and Co-Domain}
\begin{enumerate}
\item$f:A\to B$
\item$ f(x)=x^2$
\item$ f(x)= \sum_{k=1}^x k$
\item$ f \circ g ;$ $ f:B\to C ; g: A \to B$
\end{enumerate}
\textbf{Find a function with the following properites:}
\begin{enumerate}
\item$f:\mathbb{N}\to\mathbb{R}$, 1-1, not onto
\item$f:\mathbb{R}\to\mathbb{N}$, onto, not 1-1
\item$f:\mathbb{N}\to \{True, False\}$, not 1-1, onto
\item$\mathbb{N}\to\{\bullet\}$, not 1-1, onto
\end{enumerate}
\textbf{Is is possible to make $f(x)=\sqrt{x}$ a total function (i.e. a function defined across its entire Domain)? If so, how?}
\subsection{Using Function Definitions}
\begin{align*}
f(m^\curvearrowright)&=(m^\curvearrowright)+f(m)\\
f(0)&=0
\end{align*}
\textbf{Find $f(5)$, show steps}\\
\newline
\textbf{You have seen this function before, what does it usually look like?}
\newpage
\section{Inductive Proofs}
\subsection{Fill-in-Blanks}
Prove: $m(n+p)=m\cdot n + m\cdot p$
\begin{proof}
$\bullet [Basis]: 0(n+p)=0\cdot n +0\cdot p  $
\begin{align*}
0\cdot n +0\cdot p&=0+0&&\text{$0$ is an annihilator}\\
&=0&&\text{\underline{\hspace{3cm}}}\\
&=0(n+p)&&\text{$0$ is an annihilator}\\
\end{align*}
$\bullet[IH]: \exists{k} \text{\hspace{0.2cm}}s.t.\text{\hspace{0.2cm}} k(n+p)=k\cdot n+k\cdot p$\\
$\bullet[IS]$: We need to show that \underline{\hspace{3cm}}
\begin{align*}
(k^\curvearrowright)(n+p)&=(n+p)+k(n+p) &&\text{Definition of Multiplication}\\
&=\text{\underline{\hspace{3cm}}}&&\text{IH}\\
&=n+k\cdot n+p+k\cdot p &&\text{Commutativity of Addition}\\
&=(k^\curvearrowright)n+(k^\curvearrowright)p &&\text{\underline{\hspace{3cm}}}
\end{align*}
\end{proof}
\subsection{Writing Proofs}
\paragraph{Prove:}$\sum_{i=0}^n 4^i=\frac{4^{n+1}-1}{3}$ \hspace{3cm} $ \forall n\in\{x\in\mathbb{N}|x>0\}$
\end{document}