\documentclass[12pt]{article}
\usepackage{mathtools}
\usepackage{amsfonts}
\usepackage{amssymb}
\usepackage{amsthm}
\usepackage{enumitem}
\usepackage[a4paper, total={6in, 10in}]{geometry}
\author{}
\date{}
\title{SI 10: Set Theory and Logic}
\begin{document}
	\maketitle
	\textbf{{Assume that all sets mentioned here are subsets of a set $X$.}
	\section{Drawing Sets}
	\textbf Draw Diagrams for the following: }
	\begin{enumerate}
		\item $(A \cup B) \cap C \to A \cup B$
		\item $A \to (A \setminus B)$
		\item $A \to (A\setminus B) \cap B$
		\item $ (A \cup C) \setminus (A \cap C)$
		\item$ (A \cap C) \setminus (A \cup C) \to (A \setminus B) \cap B$
	\end{enumerate}
	In logic, there is the concept of negation, denoted by the operator ``$\neg$", and it is read as ``not". So ``$\neg A$" is read as ``not $A$". Now, using only previously defined operators:
	\begin{enumerate}[resume]
		\item Come up with a definition for $\neg$. In other words, $\neg A \iff ?$
	\end{enumerate}
	\section{Set Proofs}
	\textbf{Using only the set rules defined on the next page, prove the following:}
	\begin{enumerate}[resume]
		\item $(A\cup B)\cup C = A \cup (B \cup C)$
		\item $A \cup (B \cap C) = (A \cup B) \cap (A \cup C)$
	\end{enumerate}
\newpage
	\section*{Set Rules}
	\begin{enumerate}
		\item $D \cup E \subseteq F \:\: \iff \:\: D\subseteq F\text{ and } E \subseteq F$
		\item $D \subseteq E \cap F \:\: \iff \:\: D\subseteq E \text { and } D \subseteq F$
		\item $D \subseteq E \cup F \:\: \iff \:\: D \setminus E \subseteq F$
		\item $\cup$ and $\cap$ are commutative
	\end{enumerate}
\end{document}